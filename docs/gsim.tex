\documentclass{article}
%\usepackage{html,psfig}

\setlength{\textwidth}{5.5 in}
\addtolength{\oddsidemargin}{-0.5in}
\addtolength{\evensidemargin}{-0.5in}

\begin{document}
\sloppy
\title{GSIM User's Guide Version 1.1}
\author{Elliott Wolin,\\ College of William and Mary,\\ wolin@cebaf.gov}
\date{\today}
\maketitle

\begin{abstract}
I describe how to use the CLAS Geant simulation, GSIM, including how
to generate BOS/FPACK output files and how to modify the user routines
to create a custom simulation. \\
\par
The postscript version of this note resides in
%\htmladdnormallink{\$CLAS\_DOC/gsim\_userguide.ps}{ftp://ftp.cebaf.gov/pub/clas/doc/gsim_userguide.ps}.
The html version is in http://www.cebaf.gov/clas/gsim/userguide/userguide.html.
\end{abstract}

\section{Introduction}
\label{introduction}

GSIM is the official CLAS Geant simulation framework.  It consists of
a central steering and control package that calls a number of
independent detector geometry and response packages.

GSIM is modular in that it is relatively easy to replace a detector
package with another, or to add an additional detector package.

Numerous user hooks are supplied to allow users to customize the run.
Sensible defaults are provided for all options, including the user
routines.

GSIM is officially a subroutine library.  The example main programs
and executables are probably sufficient for most applications.

It is very simple to get started if default settings are acceptable.

Notes:
\begin{quote}
GSIM doesn't use patchy \\
KUIP is not used in the batch version \\
GSIM has no built-in histogram package
\end{quote}

%See the \htmlref{Reference}{references} section for a list of useful references.


\section{Quick start}
\label{quickstart}

First perform CLAS setup (e.g. on the CEBAF HP cluster):
\begin{quote}
{\tt source /clas/cebafb/u1/etc/profile/.clascshrc}
\end{quote}
(This can be done in your own .cshrc file.) Then type:
\begin{quote}
{\tt gsim\_bat}
\end{quote}
or
\begin{quote}
{\tt \$CTEST\_EXE/gsim\_bat}
\end{quote}
to start the production or test batch versions of GSIM.  The test
version is more up to date, the production version is more stable (use
the test version for now, 10/95...EJW).  After the prompt:
\begin{quote}
{\tt CLAS\_FFGO  I: enter FFREAD data cards now:}
\end{quote}
enter the FFREAD cards for the run (see GEANT manual), e.g:
\begin{quote}
{\tt trig 100} \\
{\tt stop}
\end{quote}
to generate 100 events using the default kinematics generator.  By
default, all physics processes are turned on, full showers are
generated in the calorimeter, etc.  The BOS/FPACK event
output file created is named gsimout.evt.  Note that the start counter
geometry is NOT generated by default (see htmlref{ffread}{ffread}
section if you want the start counter).

The interactive version is started with:
\begin{quote}
{\tt gsim\_int}
\end{quote}
or
\begin{quote}
{\tt \$CTEST\_EXE/gsim\_int}
\end{quote}
Enter 1 for the terminal type (assuming the DISPLAY environment
variable is set properly).  Then proceed as for the batch version.
See the GEANT manual for numerous drawing, run control, and data
display commands.

Command line switches can simplify running GSIM.  The above procedure
could be shortened to:
\begin{quote}
{\tt gsim\_bat -noffread -trig 100 }
\end{quote}
or
\begin{quote}
{\tt gsim\_int -noffread -trig 100}
\end{quote}
Finally, type:
\begin{quote}
{\tt gsim\_bat -help}
\end{quote}
to view the GSIM help file
%\htmladdnormallink{(\$CLAS\_DOC/gsim.hlp),}{ftp://ftp.cebaf.gov/pub/clas/doc/gsim.hlp}
which lists
numerous command line options, additional CLAS ffread keys, which
environment variables are used, etc.  This file will always be kept up
to date.  See also the appendices on
htmlref{command line}{commandline} flags,
htmlref{environment}{environmentvariables} variables,
htmlref{kinematics}{kinematics} generator options,
htmlref{ffread}{ffread} flags, and
htmlref{switch}{switches} usage.


\section{Customizing the run}
\label{customizing}

There are 2 ways to customize the run: run-time customization uses the
default executables; compile-time customization requires relinking your
own version of gsim.

\subsection{Run-time customization}
\label{runtime}

Run-time customization will work for you if the standard geometry is
acceptable, the built-in kinematics generators are acceptable or you
can supply a BOS/FPACK or `Liz-format' input kinematics file, and all
you want to do is turn on or off some physics processes, turn off some
detectors, modify the B field strength, etc.  Here you analyze the run
using the BOS/FPACK output file.

See the appendix on htmlref{kinematics generators}{kinematics} for
information on the built-in kinematics generators.  All possible
run-time customization options are listed in the gsim help file
%\htmladdnormallink{(\$CLAS\_DOC/gsim.hlp),}{ftp://ftp.cebaf.gov/pub/clas/doc/gsim.hlp}
or type: gsim\_bat -help), and in the CERN Geant
manual in section BASE040.  See also the appendices on
htmlref{command line}{commandline} flags,
htmlref{environment}{environmentvariables} variables,
htmlref{ffread}{ffread} flags, and
htmlref{switch}{switches} usage.


\subsection{Compile-time customization}
\label{compiletime}
Compile-time customization is required if you want to modify the
geometry, provide a custom kinematics generator, spy on the internal
workings of GSIM, fill histograms during the simulation, etc.  Linking
your own personal copy of GSIM involves use of the CLAS code
management system.  For full details see CLAS-NOTE-95-007 by Dieter
Cords.

See also the appendix on htmlref{user routines}{userroutines} for a
list of routines than can be modified by the user.

Briefly (see CLAS-NOTE-95-007), after:
\begin{quote}
{\tt setup gnu} \\
{\tt setup tcl} \\
{\tt setup cernlib}
\end{quote}
(the above can be done in your own .cshrc file).
\begin{enumerate}

\item Clone your own personal gsim tree:
\begin{quote}
{\tt cloneGsim}
\end{quote}
to create a new directory tree gsim/ in the current directory.

\item If you wish, edit gsim/gsim\_bat/hpf77.plt/imports.imp or
gsim/gsim\_int/hpf77.plt/imports.imp to modify compiler
flags, change the default link library order, etc.

\item In gsim/gsim.lib/s check out (unlocked!) and edit the guser\_xxxx
routines you want to modify, e.g.:
\begin{quote}
{\tt co guser\_book.F guser\_hist.F	}
\end{quote}
ALL routines in this directory will be compiled and linked into the
executable, so you may place additional source routines here.
\label{begstep}

\item If you wish, edit the default main programs gsim/gsim\_bat/s/gsim\_bat.F or
gsim/gsim\_int/s/gsim\_int.F.

\item In gsim/gsim\_bat/hpf77.plt or gsim/gsim\_int/hpf77.plt, type:
\begin{quote}
{\tt make depends}
\end{quote}
to generate an include file dependency list from the fortran source
files in gsim.lib/s and gsim\_bat/s or gsim\_int/s (for make).  This
must be redone only when the list of include files used in the source
files changes.

\item Type:
\begin{quote}
{\tt make RCS\_UPD=yes full}
\end{quote}
or
\begin{quote}
{\tt make\_upd full}
\end{quote}
to compile and link with the production or test libraries,
respectively.  The executables are named gsim\_bat and gsim\_int.

\item Run the executables as described in the %\htmlref{Quick Start}{quickstart} section.
\label{endstep}

\item Repeat steps \ref{begstep} to \ref{endstep} as required.

\item Type:
\begin{quote}
{\tt make clean}
\end{quote}
to clean up all object files and core files, leaving only the executables.

\end{enumerate}

\section{Help, comments, etc.}
\label{help}

See the notes in the %\htmlref{Reference}{reference} section, and read
the online help file
%\htmladdnormallink{(\$CLAS\_DOC/gsim.hlp),}{ftp://ftp.cebaf.gov/pub/clas/doc/gsim.hlp}
or type: gsim\_bat~-help).
The latter always has the most recent lists of flags, settings, etc.,
and is more up to date than this note.

Send additional questions, comments, criticisms, etc. to:
\begin{quote}
email: %\htmladdnormallink{wolin@cebaf.gov}{mailto:wolin@cebaf.gov} \\
\\
Elliott Wolin \\
College of William and Mary\\
804-221-3532
\end{quote}


\newpage
\section{References}
\label{references}

\begin{enumerate}

\item Cern GEANT manual; currently we are using version 3.21.

\item %\htmladdnormallink{\$CLAS\_DOC/gsim.hlp}{ftp://ftp.cebaf.gov/pub/clas/doc/gsim.hlp}
is the GSIM online help file.  This will
always contain the latest flags, switches, etc.

\item %\htmladdnormallink{BOS,}{ftp://ftp.cebaf.gov/pub/clas/doc/BOS.ps}
%\htmladdnormallink{FPACK,}{ftp://ftp.cebaf.gov/pub/clas/doc/FPACK.ps}
and %\htmladdnormallink{H1UTIL}{ftp://ftp.cebaf.gov/pub/clas/doc/H1UTIL.text}
manuals.

\item CERN FFREAD manual.

\item %\htmladdnormallink{\$CLAS\_WWW/gsim/STRATEGY}{http://www.cebaf.gov/clas/gsim/STRATEGY}
lists basic information on internal
details.

\item  See %\htmladdnormallink{\$CLAS\_WWW/gsim/STILLTODO}{http://www.cebaf.gov/clas/gsim/STILLTODO}
for the never-ending list of
improvments planned.

\item See %\htmladdnormallink{\$CLAS\_WWW/gsim/PROGRESS}{http://www.cebaf.gov/clas/gsim/PROGRESS}
for a list of recent
improvements; the entries are in chronological order.

\item See %\htmladdnormallink{\$CLAS\_WWW/gsim/PACKAGES}{http://www.cebaf.gov/clas/gsim/PACKAGES}
for further background information concerning individual packages.

\item %\htmladdnormallink{\$CLAS\_WWW/gsim/call\_tree.txt}{http://www.cebaf.gov/clas/gsim/call\_tree.txt}
lists the GSIM call tree.

\item %\htmladdnormallink{\$CLAS\_WWW/gsim/file\_summary.txt}{http://www.cebaf.gov/clas/gsim/file\_summary.txt}
lists the include files
used in each subroutine (indexed by subroutine name)

\item %\htmladdnormallink{\$CLAS\_WWW/gsim/inc\_summary.txt}{http://www.cebaf.gov/clas/gsim/inc\_summary.txt}
lists which files use each
include file (indexed by include file name).

\item %\htmladdnormallink{\$CLAS\_WWW/gsim/BUGS}{http://www.cebaf.gov/clas/gsim/BUGS}
lists bugs or odd features I found on different machines.

\end{enumerate}


\appendix



\newpage
\section{Command Line Flags}
\label{commandline}

The following run control command line flags are available.  Note that
command line flags always override FFREAD cards.  All command flags
must be preceded by a `-':

\par
\begin{center}
\begin{tabular}{|l|l|} \hline
flag     &   description \\ \hline
h        &   print \$CLAS\_DOC/gsim.hlp \\
help     &   print \$CLAS\_DOC/gsim.hlp \\
nogeom   &   set all geometry off by default \\
noffread &   FFREAD cards NOT read in \\
nomate   &   all materials set to VACUUM \\
nosec    &   no secondaries generated \\
nohits   &   no hits or digitizations generated \\
nodigi   &   no digitizations generated \\
nobosout &   no BOS output file created \\
nodata   &   no simulated data BOS output \\
nomcdata &   no mc data BOS output \\
trig n   &   generate n events \\
kine i   &   set kinematics generator \\ \hline
\end{tabular}
\end{center}
\par

\noindent
The following filename specification command line flags are available.

\par
\begin{center}
\begin{tabular}{|l|l|l|} \hline
flag     &  description                            & default \\ \hline
ffread   &  ffread input file name                 & read from stdin  \\
geom     &  BOS geom input file                    & not implemented yet \\
ecatten  &  EC attenuation length input file name  & \$CLAS\_PARMS/ec\_atten.dat \\
mcin     &  BOS MCIN input kinematics file         & mcin.evt \\
lizin    &  Liz format input KINE file name        & lizin.txt \\
bgrid    &  torus field input file name            & \$CLAS\_PARMS/bgrid.fpk \\
bosout   &  BOS output file name                   & gsimout.evt \\ \hline
\end{tabular}
\end{center}

\noindent
For example,
\begin{quote}
{\tt gsim\_bat -mcin myfile.evt -nosec -noffread}
\end{quote}
tells GSIM to use the file `myfile.evt' as the BOS/FPACK kinematics
input file (if EITHER the option KINE 1 is specified OR no KINE option
is specified);  to inhibit all secondary generation (i.e. no
showers, decays, etc.);  and to NOT read in any FFREAD cards.


\newpage
\section{Environment Variables}
\label{environmentvariables}

The following directory specification envinronment variables are used
by GSIM.  \par
\begin{center}
\begin{tabular}{|l|l|} \hline
Env variable  &  description \\ \hline
CLAS\_ROOT    &  location of root of CLAS code tree \\
CLAS\_PARMS   &  location of CLAS parameter and data files \\
CLAS\_DOC     &  location of CLAS documention files \\ \hline
\end{tabular}
\end{center}

\par
The following file name environment variables are used.  Note that
specification of a command line flag will override the environment
variable definition.
\par
\begin{center}
\begin{tabular}{|l|l|} \hline
Env variable  &  description \\ \hline
GSIM\_FFREAD  &  ffread input file name \\
GSIM\_GEOM    &  BOS geom input file name (not implemented yet) \\
GSIM\_ECATTEN &  EC attenuation length input file name \\
GSIM\_MCIN    &  BOS input kinematics file name \\
GSIM\_LIZIN   &  `Liz-format' input kinematics file name   \\
GSIM\_BGRID   &  torus field grid input file name      \\
GSIM\_BOSOUT  &  BOS output file name \\ \hline
\end{tabular}
\end{center}

\noindent
For example:
\begin{quote}
{\tt setenv GSIM\_FFREAD myffread.txt}
\end{quote}
sets the input FFREAD file name to `myffread.txt'.


\newpage
\section{FFREAD keys}
\label{ffread}

The following GSIM-specific FFREAD keys are defined.  See the GEANT
manual for numerous additional key definitions for controlling multiple
scattering, energy loss, etc.  See the section of
htmlref{kinematics}{kinematics} generators for more details on the
KINE option.  See the section on
htmlref{switches}{switches} for GSIM use of the SWIT key.

\par
\begin{center}
\begin{tabular}{|l|l|l|} \hline
keyname   & arguments    &  values and meaning \\ \hline
NOFIELD   &  T       	 &  turn off all B fields (torus,mini,ptg) (default F) \\ \hline
KINE      &  i       	 &  select kinematics generator (default 0) \\ \hline
MAGTYPE   &  type    	 &  $<0$ call guser\_fld to get complete field \\
          &      	 &  0 for no torus or mini (ptg if requested) \\
	  &          	 &  1 for simple analytic torus, no mini (ptg if requested)  \\
	  &          	 &  2 for CLAS torus field, no mini (ptg if requested) \\
	  &          	 &  3 for CLAS torus + mini (ptg illegal) (default) \\
	  &          	 &  4 for mini-torus only (ptg illegal) \\
          &  field   	 &  analytic torus field strength (kG,real) \\ \hline
MAGSCALE  &  2f   	 &  scale factors for torus and mini fields \\ \hline
FIELD     &  type    	 &  1 for Runge-Kutta \\
          &          	 &  2 for Helix (default)\\
          &  max     	 &  maximum field strength (default 30 kG) \\ \hline
NSTEPMAX  &  max     	 &  max steps per track (default 10000) \\ \hline
SLENGMAX  &  max     	 &  max track length (default 20000 cm)\\ \hline
ZMIN	  &  min     	 &  min track Z coord (default -300 cm) \\ \hline
ZMAX	  &  max     	 &  max track Z coord (default 30000 cm)\\ \hline
HELIUM	  &  T     	 &  Replace all Air with He in CLAS detector\\
	  &  F     	 &  Use Air (default) \\ \hline
IFLGK     &  flag     	 &  0 to track, do not store secondaries (default) \\
          &  	     	 &  1 to track, store secondaries \\ \hline
DRIFT     &dist\_to\_time&  DC conversion factor from dist to time (default 400.0) \\ \hline
DCRES     &  6f          &  DC resolution for the 6 superlayers (default 0.020 cm) \\ \hline
ATLEN     &  length    	 &  $=0.$ no attenuation in EC\\
          &          	 &  $<0.$ read EC atten lengths from \$CLAS\_PARMS/ec\_atten.dat (default)\\
          &       	 &  $>0.$ use length for all EC strips \\ \hline
POISS     &  f           &  Poisson parameter for EC photon generation (default 3.5) \\ \hline
ECTDC     &  i           &  EC conversion factor from tdc to channel (default 20) \\ \hline
PTGANGLE  &  angle       &  angle of polarized target (default 0. degrees) \\ \hline
PTGIFIELD &  type        &  1 for Runge-Kutta \\
          &              &  2 for Helix (default) \\ \hline
PTGFIELDM &  max         &  max field for polarized target (default 50.0 kG, 0. to turn off) \\ \hline
PTGSCALE  &  scale       &  scale factor for polarized target field (default 1.) \\ \hline
\end{tabular}
\end{center}


\par
\noindent
The tracking media cuts for all detectors can be set with:
\par
\begin{center}
\begin{tabular}{|l|l|l|} \hline
keyname   & arguments    &  values and meaning \\ \hline
CCCUTS    &  5f    	 &  set cutgam,cutele,cutneu,cuthad,cutmuo for CC \\ \hline
DCCUTS    &  5f    	 &  set cutgam,cutele,cutneu,cuthad,cutmuo for DC \\ \hline
ECCUTS    &  5f    	 &  set cutgam,cutele,cutneu,cuthad,cutmuo for EC \\ \hline
SCCUTS    &  5f    	 &  set cutgam,cutele,cutneu,cuthad,cutmuo for SC \\ \hline
STCUTS    &  5f    	 &  set cutgam,cutele,cutneu,cuthad,cutmuo for ST \\ \hline
CUTS      &  5f    	 &  set cutgam,cutele,cutneu,cuthad,cutmuo for everything else \\ \hline
\end{tabular}
\end{center}

\par
\noindent
The following keys take a list of inert material and detector names as arguments.
\par
\begin{center}
\begin{tabular}{|l|l|l|} \hline
keyname   &  arguments                    &  meaning \\ \hline
NOGEOM    &  detectors or inert materials &  don't generate geometry,hits,etc. \\
          &                               &  'ALL' means turn off all geom by default \\ \hline
GEOM      &  detectors or inert materials &  generate geometry if not on by default, \\
          &                               &  or if turned off with -nogeom or NOGEOM 'ALL' \\  \hline
NOMATE    &  detectors or inert materials &  set materials to vacuum \\ \hline
NOSEC     &  detectors                    &  no secondaries generated \\  \hline
NOHITS    &  detectors                    &  no hits,digitizations,BOS output generated \\  \hline
NODIGI    &  detectors                    &  no digitizations generated \\  \hline
NOBOSOUT  &  detectors                    &  no BOS output file generated \\  \hline
NODATA    &  detectors                    &  no simulated data BOS output banks generated \\  \hline
NOMCDATA  &  detectors                    &  no mc data BOS output banks generated \\ \hline
\end{tabular}
\end{center}
\par

\noindent
Legal detector names are:
\begin{quote}
'ALL' 'CC' 'DC' 'EC' 'EC1' 'SC' 'ST'
\end{quote}

\noindent
Legal inert material names are:
\begin{quote}
'TORU' 'MINI' 'FOIL' 'PTG' 'OTHE'
\end{quote}

\noindent
Geometry NOT on by default (use GEOM card to turn on):
\begin{quote}
'ST' 'PTG'
\end{quote}

\noindent
Note that more than one detector name or inert material name can be
listed on the same line.  A second card with the same key overrides
the first one.

For example:
\begin{quote}
{\tt list} \\
{\tt kine    1} \\
{\tt magtype 0} \\
{\tt nodigi 'ALL'} \\
{\tt nogeom 'SC'} \\
{\tt nohits 'EC' 'CC'}
\end{quote}
causes GSIM to: read the kinematics information from a BOS/FPACK
kinematics file (filename either default mcin.evt or set with -mcin
command line flag); turn off the magnetic field; inhibit all
digitizations; not generate the SC
geometry, hits, etc; and not generate hits for EC and CC.

The `list' key causes the FFREAD cards to be listed on stdout.  See
the CERN FFREAD manual for more options.


\newpage
\section{Built-in kinematics generators}
\label{kinematics}

The default is KINE 0.  Possibilities are:

\par
\begin{center}
\begin{tabular}{|l|l|} \hline
kine option  &  description \\ \hline
0 &  Call guser\_kine.  The default guser\_kine generates 4 random \\
  &  tracks with $0.8 < p < 4.0\ gev$, no physics, no momentum\\
  &  conservation.  The tracks are $e^-, \pi^-, \pi^0, \pi^+$. \\ \hline
1 &  GSIM gets the kinematics from MCVX and MCTK banks in the \\
  &  BOS/FPACK file specified by the -mcin command line flag or \\
  &  by the environment variable GSIM\_MCIN. \\ \hline
2 &  GSIM gets kinematics from the "Liz-format" file specified \\
  &  by the -lizin command flag or by the environment variable  \\
  &  GSIM\_LIZIN. Contact me or Liz for the format of this file. \\ \hline
3 &  GSIM generates single track events \\
  &  additional args:  geant\_id,pmin,pmax,thetamin,thetamax,phimin,phimax \\
  &  Units are gev,degrees, defaults are e-, 0.8,3.5, 30.,40., -10.,10. \\ \hline
\end{tabular}
\end{center}
\par

\noindent
I will add additional Generators if needed.

Other options:
\begin{itemize}
\item you can write your own guser\_kine and recompile and
relink GSIM
\item your favorite kinematics generator can write the
kinematics information into MCVX and MCTK banks in a BOS/FPACK file,
which can be read in with KINE 1 (set the file name with the -mcin
command line flag)
\item your favorite generator can write the kinematics
inforation in `Liz-format', which can be read in with KINE 2 (set the
file name with the -lizin command line flag).
\end{itemize}


\newpage
\section{Switch usage}
\label{switches}

The following switches are defined for use with the SWIT key in the
FFREAD cards:

\par
\begin{center}
\begin{tabular}{|l|l|l|} \hline

Switch &  value   &   purpose \\ \hline
1      &  1       &   print tracking step information\\
       &  2       &   store and plot track points (int vsn only) \\
       &  3       &   print, store, and plot track points (int vsn only) \\ \hline
2      &  xxxxx1  &   list material definitions \\
       &  xxxx1x  &   list media definitions \\
       &  xxx1xx  &   list volume definitions \\
       &  xx1xxx  &   list set definitions \\ \hline
\end{tabular}
\end{center}
\par

\noindent
For example, the FFREAD card:
\begin{quote}
{\tt swit 1 101}
\end{quote}
causes GSIM to print all tracking step information (a lot of
output!) and material and volume definitions to stdout.



\newpage
\section{Signals}
\label{signals}

The following Unix signals are used by GSIM.
\par
\begin{center}
\begin{tabular}{|l|c|l|} \hline
signal   &  number  &  usage \\ \hline
SIGTERM  &   15     &  graceful termination at end of current event \\ \hline
\end{tabular}
\end{center}

\noindent
For example:
\begin{quote}
{\tt kill -TERM pid}
\end{quote}
or
\begin{quote}
{\tt kill -15 pid}
\end{quote}
(where pid is the process id of a gsim batch process) stops the
run at the end of the current event, writes out all data, closes
files, etc.

\newpage
\section{User routines}
\label{userroutines}

The user can modify any of a number of guser\_xxxx routines.  Sensible
defaults are provided in all cases:

\par
\begin{center}
\begin{tabular}{|l|l|} \hline
file          & purpose \\ \hline
guser\_ffky.F & custom FFREAD key definitions \\
guser\_init.F & user initializations before geometry definition \\
guser\_part.F & define extra particles \\
guser\_geom.F & define extra geometry \\
guser\_phys.F & custom physics modifications \\
guser\_book.F *** & histogram booking \\
guser\_beg.F  & last initializations before run begins \\
guser\_trev.F & control event tracking \\
guser\_kine.F & custom event kinematics definition (if KINE le 0) \\
guser\_trak.F & control individual track tracking \\
guser\_skip.F & control which tracks to skip \\
guser\_fld.F  & user-supplied B field (when MAGTYPE lt 0) \\
guser\_step.F & called by gustep at each tracking step \\
guser\_digi.F & custom digitizations \\
guser\_hist.F *** & histogram filling at end of event \\
guser\_bos.F  & fill user BOS banks at end of event \\
guser\_last.F & called at end of run \\ \hline
\end{tabular}
\end{center}
\par

The starred routines are the ones most often modified.





\end{document}